\documentclass[12pt,a4paper,notitlepage]{report}
%\usepackage[showframe=false]{geometry}
%\usepackage[francais]{babel}
\usepackage[utf8]{inputenc}
%\usepackage{kpfonts}
\usepackage{amsmath}
\usepackage{mathtools}
\usepackage{listings}
\usepackage[showframe=false,tmargin=2.5cm,bmargin=2.5cm,lmargin=2.5cm,rmargin=2.5cm]{geometry}
%\usepackage[T1]{fontenc}

\lstset{
  basicstyle=\ttfamily,
  mathescape
}
\usepackage{amssymb}
\newcommand{\R}{\mathbb{R}}
\begin{document}
\begin{large}
\textbf{Mohamed Attia}

M2-MMSI
\end{large}

\begin{center}
\begin{huge}
\vspace{8cm}
\textbf{Compte rendu }\\

\vspace{1cm}

Imagerie et applications professionnelles
\end{huge}

\begin{LARGE}

\vspace{0.5cm} MA0946
\end{LARGE}

\vspace{8cm}
\begin{large}
05 Juin 2018

\end{large}
\end{center}

\newpage
\subsection*{Objectifs :}
Le travail consiste à 
\begin{enumerate}
\item Programmer une fonction de comparaison d'une image à une base de données. 
\item Programmer sur Matlab différents descripteurs de couleurs.
\item Intégrer les descripteurs dans une interface graphique
\end{enumerate}

\section*{1.Algorithme de comparaison:}
\subsection*{Base de données image:}
On choisit une base de données contenant 50 images divisées en 5 catégories (Fusées, Cubes de Rubick, Voitures, Planètes et Nature) comme suit:

\begin{center}
\includegraphics[scale=0.5]{1.png}\\
Figure 1: Base de données
\end{center}
On utilise le programme \texttt{stockage.m} pour permettre de transformer la base de données en une variable utilisable par Matlab.
\subsection*{Distances:}
On utiliser quatre méthodes pour calculer la distance entre les histogrammes des images:
\begin{itemize}
	\item $d_{L1}(V_1,V_2)=\displaystyle\sum_{i=1}^{256} |V_{1i}-V_{2i}|$
	\item $d_{L2}(V_1,V_2)=\displaystyle \sqrt{\sum_{i=1}^{256} (V_{1i}-V_{2i})^2}$
	\item $d_{L3}(V_1,V_2)=\displaystyle\sup_{i=1,256} |V_{1i}-V_{2i}|$
	\item $d_{L4}(V_1,V_2)=\displaystyle\sum_{i=1}^{256} \frac{(V_{1i}-V_{2i})^2}{(V_{1i}+V_{2i})^2}$

\end{itemize}
\subsection*{Programme principal:}
Dans le fichier \texttt{main.m} on peut saisir la base de données images. Ensuite on extrait les histogrammes des images et on compare une image extérieure à toute les images de la base de données avec les différentes méthodes.
Exemple d'exécution:
\begin{center}
\includegraphics[scale=0.5]{2.png}\\
Figure 2: Résultat de comparaison
\end{center}
\section*{2.Descripteurs couleurs:}
Dans cette partie on cherche à réaliser plusieurs descripteur 
\begin{itemize}
\item Premier descripteur: Histogramme de niveau de gris:\\
La fonction \texttt{histoNG.m} transforme une image RGB en nuance de gris pour en extraire un vecteur représentant le descripteur.
\item Deuxième descripteur: Histogramme à trois couleurs:\\
La fonction \texttt{histoRGB.m} permet d'extraire à partir d'une image RGB un vecteur représentant l'histogramme de chaque plan couleur.
\item Troisième descripteur: RGB indexé:\\
La fonction \texttt{histoRGBI.m} permet d'extraire à partir d'une image RGB indexée un vecteur représentant l'image indexée ainsi que les différents valeurs de quantification des couleurs.
\item Quatrième descripteur: HSV indexé:\\
La fonction \texttt{histoHSVI.m} permet d'extraire à partir d'une image HSV indexée un vecteur représentant l'image indexée ainsi que les différents valeurs de quantification des couleurs.
\item Cinquième descripteur: Histogramme Hue:\\
La fonction \texttt{histohue.m} permet d'extraire à partir d'une image HSV un vecteur représentant le histogramme de Hue de l'image puis le représenter sur un cercle.
\item Sixième descripteur Moment d'ordre 1 (Moyenne) et Moment d'ordre 2  (Variance):\\
Les fonctions \texttt{Moment1.m} et \texttt{Moment2.m} permettent de calculer d'une image en nuance de gris un vecteur représentant la moyenne et la variance grâce à une quantification sur les lignes et les colonnes.
\end{itemize}

\section*{3.Scalable Color Descriptor:}
Le Scalable Color Descriptor est un histogramme dans l'espace colorimétrique HSV. Il est utile pour la comparaison image-image et la recherche basée sur la fonction de couleur.

Pour créer ce descripteur on commence avec une image HSV puis en la transforme en une image HSV indexée. Ensuite, à partir de l'histogramme de cette image on crée un vecteurs ayant la moitié de la dimension du premier histogramme. On continue ainsi jusqu'à avoir deux vecteurs : un pour la somme des pixels 2 à 2 et un autre pour la différence des pixel 2 à 2:\\
Histogramme HSV 256\\
$ h_{0}$
\begin{tabular}{|c|c|c|}
	\hline 1&$\cdots$&256 \\ \hline
\end{tabular}
\begin{tabular}{l l l}
	$\Longrightarrow$ $ h_{1+}$

\begin{tabular}{|c|c|c|}
	\hline 1&$\cdots$&128 \\ \hline
\end{tabular} 
&
 $\Longrightarrow$ $ h_{2+}$
\begin{tabular}{|c|c|c|}
	\hline 1&$\cdots$&64 \\ \hline
\end{tabular}
&
$\Longrightarrow$ $ h_{3+}$
\begin{tabular}{|c|c|c|}
	\hline 1&$\cdots$&32  \\ \hline
\end{tabular}
\\
$\Longrightarrow$ $h_{1-}$
\begin{tabular}{|c|c|c|}
	\hline 1&$\cdots$&128 \\ \hline
\end{tabular}

&
$\Longrightarrow$ $h_{2-}$
\begin{tabular}{|c|c|c|}
	\hline 1&$\cdots$&64 \\ \hline
\end{tabular}
&
$\Longrightarrow$ $h_{3-}$
\begin{tabular}{|c|c|c|}
	\hline 1&$\cdots$&32 \\ \hline
\end{tabular}
\end{tabular}

Finalement on aura:\\
Le vecteur somme: $h_{s}$
\begin{tabular}{|c|c|c|}
	\hline 1&$\cdots$&16 \\ \hline
\end{tabular}
et le vecteur différence: $h_{d}$
\begin{tabular}{|c|c|c|}
	\hline 1&$\cdots$&16 \\ \hline
\end{tabular}
Le tableau suivant résume la quantification de chaque plan de l'espace calorimétrique HSV:
\begin{center}
\begin{tabular}{|c|c|c|c|c|c|}
	\hline 
	& 256 & 128 & 64 & 32 & 16 \\ \hline
	H& 16 & 8 & 8 & 8 & 4\\ \hline
	S& 4 & 4 & 2 & 2 & 2\\ \hline
	V& 4 & 4 & 4 & 2 & 2\\ \hline
\end{tabular}
\end{center}

Le descripteur final est constitué de la concaténation de tous les histogrammes.


\section*{4.Dominant Color Descriptor:}
Le descripteur de couleur dominante est un descripteur compact qui nous permettra de représenter efficacement les couleurs dominantes sur une image.

On commence par l'utilisation de la méthode de classification Kmeans à N nombres de classes (ici une choisit 3) sur des images en espace couleurs HSV, LAB ou RGB.

On définit pour chaque centre de classe les paramètres associés c'est à dire:
\begin{itemize}
	\item	pour $C_1$: ($H_{C1},S_{C1},V_{C1}$) la probabilité $p_1$ et la variance $V_1$
	\item pour $C_2$: ($H_{C2},S_{C2},V_{C2}$) la probabilité $p_2$ et la variance $V_2$
	\item pour $C_3$: ($H_{C3},S_{C3},V_{C3}$) la probabilité $p_3$ et la variance $V_3$
\end{itemize}  
où p: probabilité d'appartenance à une classe=$\frac{\text{cardinalité}}{\text{taille de l'image}}$

\begin{itemize}
	\item	Premier descripteur 
	\begin{tabular}{|c|c|c|c|c|c|}
		\hline 
		$C_1$&$p_1$ & $C_2$ & $p_2$ & $C_3$ & $p_3$ \\ \hline
	\end{tabular}
		\item	Deuxième descripteur 
	\begin{tabular}{|c|c|c|c|c|c|c|c|c|}
		\hline 
		$C_1$&$p_1$ & $V_1$ & $C_2$ & $p_2$ & $V_2$ & $C_3$ & $p_3$ & $V_3$\\ \hline
	\end{tabular}
		\item	Troisième descripteur 
	\begin{tabular}{|c|c|c|c|c|c|c|c|c|c|}
		\hline 
		$C_1$&$p_1$ & $V_1$ & $C_2$ & $p_2$ & $V_2$ & $C_3$ & $p_3$ & $V_3$ & $S$\\ \hline
	\end{tabular}
		\item	Quatrième descripteur 
	\begin{tabular}{|c|c|c|c|c|c|c|c|c|c|c|c|}
		\hline 
		$C_1$&$p_1$ & $V_1$ & $S_1$ & $C_2$ & $p_2$ & $V_2$ & $S_2$ & $C_3$ & $p_3$ & $V_3$ & $S_3$\\ \hline
	\end{tabular}
\end{itemize}
Calcul de S:\\


$$S=p_1S_1+p_2S_2+p_3S_3$$
$$S_1=\frac{N_\text{composantes connexes}}{N_\text{pixels}}$$
c'est la moyenne de taille des parties connexes (spatialité)


L'algorithme prend deux paramètres en entrée l'image 3D et le nombre de couleur dominantes


\end{document}